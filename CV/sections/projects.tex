\documentclass[letter,10pt]{article}
\usepackage{TLCresume}
\usepackage{hyperref}
\begin{document}
	
	%====================
	% EXPERIENCE A
	%====================
	\subsection{Sizing of Charging Stations Co-located with Solar Panel and Battery Storage}
	\subtext{A business project in collaboration with \textbf{ATCO Electric }company, Alberta, Canada}
	\begin{zitemize}
		\item Held several business meetings since the start of the project and analyze the problem in depth.
		\item Devleloped a problem formulation and communicated with the business team making sure the problem is properly translated into algorithm and math
		\item The project was then migrated into a programming in Python
		\item Several edge cases were considered and tested
		\item The final version of the work was verified based on the real data provided by ATCO company
		\item The report was presented as a whitepaper to the industry partners (Alberta power industry simposuim)
	\end{zitemize}
	\subsection{Reddit Clone Web Application}
	\subtext{Web Application with Angular}
	\begin{zitemize}
		\item Developed an application that allows users to post an article and a system is available for upvoting/downvoting
		\item The code follows ES6 developed using TypeScript
	\end{zitemize}

	\subsection{Breast cancer diagnosis using Machine Learning}
	\subtext{Supervised Machine Learning using classification methods from Scikit-learn python}
	\begin{zitemize}
		\item Implemented 7 different classification ML models to compare diagnostic accuracy.
	\end{zitemize}	
	\subsection{Applied ML using real data}
	\begin{zitemize}
		\item Predict the salary of a company given some features in a real dataset using regression models from scikit-learn library.
		\item Develop an API to train a machine learning model to predict the job positions of a company based on available features in their dataset using decision tree regression model from scikit-learn Python.
	\end{zitemize}

	\subsection{C Coding in Linux}
	\subtext{Operating Systems}
	\begin{zitemize}
		\item Developed a C programming code that shows the virtual memory allocation of a running process in Linux. [\href{https://github.com/Hadi2525/vmemory_maps}{C Code}]
		\item I wrote a C code to model the TCP communication between a server and several clients. This communication achieved via local and external IP addresses.
		\item Working with Valgrind in C/Linux.
	\end{zitemize}

	\subsection{Simulator for Queueing systems [\href{https://github.com/Hadi2525/queueing_theory}{Python Code}]}
	\subtext{Computer Networks \& Performance}
	\begin{zitemize}
		\item I developed an algorithm using various data structures including doubly linked lists, tensors, hash tables to model a queueing system
		\item An object-oriented programming API was developed to represent users and servers.
		\item I improved the coding so that it would have the best time complexity 
	\end{zitemize}
	
	
\end{document}